	\section{Goal}
\begin{tabular}{|p{14cm}|}
	\hline
	\\
	\textbf{Purpose} \\
	To conduct a comprehensive analysis of historical logistics and supply chain data to gain a deep understanding of the trends and patterns that have evolved over time. \\
	\\
	\textbf{Perspective} \\
	Assess the historical logistics and supply chain data from the viewpoint of logistics and supply chain management. \\
	\\
	\textbf{Environment} \\
	Within the context of developing the Logistics and Supply Chain Analytics Solution (METRICSTICS). \\
	\\
	\hline
\end{tabular}

\section{SMART Principles}
The goal is considered SMART (Specific, Measurable, Attainable, Realistic, and Timely) for the following reasons:
\begin{itemize}
	\item \textbf{Specific:} The METRICSTICS program is particularly specific in that it specifically defines the goal of examining historical logistics and supply chain data to get a thorough understanding of trends and patterns within the area. This focus enables us to give specific and meaningful information to logistics and supply chain managers, allowing informed decision-making.
	
	\item \textbf{Measurable:} The project's success is measurable through the formulation of key performance indicators and questions. These indicators could include measures of annual logistics cost reduction, inventory turnover rates, on-time delivery performance, and supply chain cycle time reduction. We can assess the influence of METRICSTICS by quantifying these aspects.
	
	\item \textbf{Attainable:} The aim is Attainable, as the application of METRICSTICS allows for the examination of past logistics and supply chain data. METRICSTICS equips logistics professionals with the necessary tools and infrastructure to collect, process, and analyze pertinent data, ultimately providing them with important insights.
	
	\item \textbf{Realistic:} The goal of METRICSTICS is realistic in relation to the jobs and responsibilities of logistics and supply chain management. It allows them to obtain a thorough understanding of previous performance, optimize logistics and supply chain strategies, and set realistic goals for future improvements. As a result, METRICSTICS skills are perfectly aligned with the requirements of logistics experts.
	
	\item \textbf{Timely:} The project is Timely since it focuses on the study of historical logistics and supply chain data spanning certain time periods, ensuring a well-defined timeframe for completing analyses. This approach ensures that we stay on track to achieve our goal within the designated timeframe and budget.
\end{itemize}
In brief, METRICSTICS follows the SMART principles of being Specific, Measurable, Attainable, Realistic, and Timely. These criteria lead our efforts to provide an analytics solution that is well-defined, quantifiable, realistic, applicable, and time-bound for logistics and supply chain professionals.

\section{Question And Metric}
\begin{itemize}[align=left, left=0em,labelsep=0em]
	\item[\textbullet\ ] \textbf{Question 1:} What is the average lead time for procurement of raw materials on a monthly and quarterly basis? 
	\begin{itemize}[align=left, left=0em,labelsep=0em]
		\item[] \textbf{Metric:}
		\begin{enumerate}[label={}, left=0em, labelwidth=1em, labelsep=0em, align=left]
			\item M1. Average monthly lead time
			\item M2. Average quarterly lead time
		\end{enumerate}
		\item[] \textbf{Mechanism:}
		\begin{enumerate}[label={}, left=0em, labelwidth=1em, labelsep=0em, align=left]
			\item[i.] Owner = Procurement Managers
			\item[ii.] Frequency Collected = Following procurement data entry
			\item[iii.] Frequency Reported = Monthly and Quarterly
		\end{enumerate}
	\end{itemize}
	
	\item[\textbullet\ ] \textbf{Question 2:} What is the percentage change in shipping costs for each shipping method, both monthly and quarterly? 
	\begin{itemize}[align=left, left=0em,labelsep=0em]
		\item[] \textbf{Metric:}
		\begin{enumerate}[label={}, left=0em, labelwidth=1em, labelsep=0em, align=left]
			\item M3. Maximum monthly shipping cost increase percentage
			\item M4. Minimum monthly shipping cost decrease percentage
			\item M5. Maximum quarterly shipping cost increase percentage
			\item M6. Minimum quarterly shipping cost decrease percentage
		\end{enumerate}
		\item[] \textbf{Mechanism:}
		\begin{enumerate}[label={}, left=0em, labelwidth=1em, labelsep=0em, align=left]
			\item[i.] Owner = Logistics Managers
			\item[ii.] Frequency Collected = Following shipping cost data update
			\item[iii.] Frequency Reported = Monthly and Quarterly
		\end{enumerate}
	\end{itemize}
	
	\item[\textbullet\ ] \textbf{Question 3:} How do we identify fluctuations in warehouse storage space utilization on a monthly and quarterly basis? 
	\begin{itemize}[align=left, left=0em,labelsep=0em]
		\item[] \textbf{Metric:}
		\begin{enumerate}[label={}, left=0em, labelwidth=1em, labelsep=0em, align=left]
			\item M7. Calculate the Monthly Average Storage Utilization Rate
			\item M8. Calculate the Quarterly Average Storage Utilization Rate
		\end{enumerate}
		\item[] \textbf{Mechanism:}
		\begin{enumerate}[label={}, left=0em, labelwidth=1em, labelsep=0em, align=left]
			\item[i.] Owner = Warehouse Managers
			\item[ii.] Frequency Collected = Following warehouse storage data analysis
			\item[iii.] Frequency Reported = Monthly and Quarterly
		\end{enumerate}
	\end{itemize}
	
	\item[\textbullet\ ] \textbf{Question 4:} Which month and quarter experienced the most significant inventory turnover rate change over the year? 
	\begin{itemize}[align=left, left=0em,labelsep=0em]
		\item[] \textbf{Metric:}
		\begin{enumerate}[label={}, left=0em, labelwidth=1em, labelsep=0em, align=left]
			\item M9. Standard deviation of monthly inventory turnover rates
		\end{enumerate}
		\item[] \textbf{Mechanism:}
		\begin{enumerate}[label={}, left=0em, labelwidth=1em, labelsep=0em, align=left]
			\item[i.] Owner = Inventory Managers
			\item[ii.] Frequency Collected = Following inventory turnover rate analysis
			\item[iii.] Frequency Reported = Monthly and Quarterly
		\end{enumerate}
	\end{itemize}
	
	\item[\textbullet\ ] \textbf{Question 5:} Which month and quarter experienced the most significant supplier delivery time changes over the year? 
	\begin{itemize}[align=left, left=0em,labelsep=0em]
		\item[] \textbf{Metric:}
		\begin{enumerate}[label={}, left=0em, labelwidth=1em, labelsep=0em, align=left]
			\item M10. Standard deviation of monthly supplier delivery times
		\end{enumerate}
		\item[] \textbf{Mechanism:}
		\begin{enumerate}[label={}, left=0em, labelwidth=1em, labelsep=0em, align=left]
			\item[i.] Owner = Procurement Managers
			\item[ii.] Frequency Collected = Following supplier delivery time analysis
			\item[iii.] Frequency Reported = Monthly and Quarterly
		\end{enumerate}
	\end{itemize}
	
	\item[\textbullet\ ] \textbf{Question 6:} How can we identify the top 10 suppliers with the most consistent on-time deliveries on a monthly and quarterly basis? 
	\begin{itemize}[align=left, left=0em,labelsep=0em]
		\item[] \textbf{Metric:}
		\begin{enumerate}[label={}, left=0em, labelwidth=1em, labelsep=0em, align=left]
			\item M11. Count the number of times each supplier has on-time deliveries in a month and quarter
			\item M12. Sort suppliers by their on-time delivery frequency
		\end{enumerate}
		\item[] \textbf{Mechanism:}
		\begin{enumerate}[label={}, left=0em, labelwidth=1em, labelsep=0em, align=left]
			\item[i.] Owner = Procurement Managers
			\item[ii.] Frequency Collected = Following supplier performance data collection
			\item[iii.] Frequency Reported = Monthly and Quarterly
		\end{enumerate}
	\end{itemize}
	
	\item[\textbullet\ ] \textbf{Question 7:} What is the most commonly used transportation route for shipments on a monthly basis? 
	\begin{itemize}[align=left, left=0em,labelsep=0em]
		\item[] \textbf{Metric:}
		\begin{enumerate}[label={}, left=0em, labelwidth=1em, labelsep=0em, align=left]
			\item M13. Count the number of shipments on each route in a month and find the route with the highest count.
		\end{enumerate}
		\item[] \textbf{Mechanism:}
		\begin{enumerate}[label={}, left=0em, labelwidth=1em, labelsep=0em, align=left]
			\item[i.] Owner = Logistics Managers
			\item[ii.] Frequency Collected = Following shipment data analysis
			\item[iii.] Frequency Reported = Monthly
		\end{enumerate}
	\end{itemize}
	
	\item[\textbullet\ ] \textbf{Question 8:} In which region do we experience the highest number of shipping delays on a yearly basis? 
	\begin{itemize}[align=left, left=0em,labelsep=0em]
		\item[] \textbf{Metric:}
		\begin{enumerate}[label={}, left=0em, labelwidth=1em, labelsep=0em, align=left]
			\item M14. Count the number of shipping delays by region for the entire year and find the region with the highest count.
		\end{enumerate}
		\item[] \textbf{Mechanism:}
		\begin{enumerate}[label={}, left=0em, labelwidth=1em, labelsep=0em, align=left]
			\item[i.] Owner = Logistics Managers
			\item[ii.] Frequency Collected = Following yearly shipping delay analysis
			\item[iii.] Frequency Reported = Yearly
		\end{enumerate}
	\end{itemize}
	
	\item[\textbullet\ ] \textbf{Question 9:} How can we determine the day of the year with the highest order fulfillment activity? 
	\begin{itemize}[align=left, left=0em,labelsep=0em]
		\item[] \textbf{Metric:}
		\begin{enumerate}[label={}, left=0em, labelwidth=1em, labelsep=0em, align=left]
			\item M15. Count the number of order fulfillments each day in a year and find the day with the highest count.
		\end{enumerate}
		\item[] \textbf{Mechanism:}
		\begin{enumerate}[label={}, left=0em, labelwidth=1em, labelsep=0em, align=left]
			\item[i.] Owner = Warehouse Managers
			\item[ii.] Frequency Collected = Following yearly order fulfillment analysis
			\item[iii.] Frequency Reported = Yearly
		\end{enumerate}
	\end{itemize}
	
	\item[\textbullet\ ] \textbf{Question 10:} What is the average transit time for shipped products for each carrier on a monthly and quarterly basis? 
	\begin{itemize}[align=left, left=0em,labelsep=0em]
		\item[] \textbf{Metric:}
		\begin{enumerate}[label={}, left=0em, labelwidth=1em, labelsep=0em, align=left]
			\item M16. Average monthly transit time by carrier
			\item M17. Average quarterly transit time by carrier
		\end{enumerate}
		\item[] \textbf{Mechanism:}
		\begin{enumerate}[label={}, left=0em, labelwidth=1em, labelsep=0em, align=left]
			\item[i.] Owner = Logistics Managers
			\item[ii.] Frequency Collected = Following shipping data entry
			\item[iii.] Frequency Reported = Monthly and Quarterly
		\end{enumerate}
	\end{itemize}
\end{itemize}
